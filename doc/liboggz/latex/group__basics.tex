\section{Ogg basics}
\label{group__basics}\index{Ogg basics@{Ogg basics}}
\subsection{Scope}\label{Scope}
This section provides a minimal introduction to Ogg concepts, covering only that which is required to use liboggz.

For more detailed information, see the {\tt Ogg} homepage or IETF {\tt RFC 3533} {\em The Ogg File Format version 0\/}.\subsection{Terminology}\label{Terminology}
The monospace text below is quoted directly from RFC 3533. For each concept introduced, tips related to liboggz are provided in bullet points.\subsubsection{Physical and Logical Bitstreams}\label{bitstreams}
The raw data of an Ogg stream, as read directly from a file or network socket, is called a {\em physical bitstream\/}.

\small\begin{alltt}
   The result of an Ogg encapsulation is called the "Physical (Ogg)
   Bitstream".  It encapsulates one or several encoder-created
   bitstreams, which are called "Logical Bitstreams".  A logical
   bitstream, provided to the Ogg encapsulation process, has a
   structure, i.e., it is split up into a sequence of so-called
   "Packets".  The packets are created by the encoder of that logical
   bitstream and represent meaningful entities for that encoder only
   (e.g., an uncompressed stream may use video frames as packets).
\end{alltt}\normalsize 
\subsubsection{Packets and Pages}\label{pages}
Within the Ogg format, packets are written into {\em pages\/}. You can think of pages like pages in a book, and packets as items of the actual text. Consider, for example, individual poems or short stories as the packets. Pages are of course all the same size, and a few very short packets could be written into a single page. On the other hand, a very long packet will use many pages.

\begin{itemize}
\item liboggz handles the details of writing packets into pages, and of reading packets from pages; your application deals only with {\tt ogg\_\-packet} structures.\item Each {\tt ogg\_\-packet} structure contains a block of data and its length in bytes, plus other information related to the stream structure as explained below.\end{itemize}
\subsubsection{serialno}\label{serialno}
Each logical bitstream is uniquely identified by a serial number or {\em serialno\/}.

\begin{itemize}
\item Packets are always associated with a {\em serialno\/}. This is not actually part of the {\tt ogg\_\-packet} structure, so wherever you see an {\tt ogg\_\-packet} in the liboggz API, you will see an accompanying {\em serialno\/}.\end{itemize}


\small\begin{alltt}
   This unique serial number is created randomly and does not have any
   connection to the content or encoder of the logical bitstream it
   represents.
\end{alltt}\normalsize 


\begin{itemize}
\item Use {\bf oggz\_\-serialno\_\-new()}{\rm (p.\,\pageref{oggz_8h_a25})} to generate a new serial number for each logical bitstream you write.\item Use an {\bf Oggz\-Table }{\rm (p.\,\pageref{oggz__table_8h})}, keyed by {\em serialno\/}, to store and retrieve data related to each logical bitstream.\end{itemize}
\subsubsection{b\_\-o\_\-s and e\_\-o\_\-s}\label{boseos}
\small\begin{alltt}
   bos page: The initial page (beginning of stream) of a logical
      bitstream which contains information to identify the codec type
      and other decoding-relevant information.\end{alltt}\normalsize 


\small\begin{alltt}   eos page: The final page (end of stream) of a logical bitstream.
\end{alltt}\normalsize 


\begin{itemize}
\item Every {\tt ogg\_\-packet} contains {\em b\_\-o\_\-s\/} and {\em e\_\-o\_\-s\/} flags. Of course each of these will be set only once per logical bitstream. See the Structuring section below for rules on setting {\em b\_\-o\_\-s\/} and {\em e\_\-o\_\-s\/} when interleaving logical bitstreams.\item This documentation will refer to {\em bos\/} and {\em eos\/} {\em packets\/} (not {\em pages\/}) as that is more closely represented by the API. The {\em bos\/} packet is the only packet on the {\em bos\/} page, and the {\em eos\/} packet is the last packet on the {\em eos\/} page.\end{itemize}
\subsubsection{granulepos}\label{granulepos}
\small\begin{alltt}
   granule position: An increasing position number for a specific
      logical bitstream stored in the page header.  Its meaning is
      dependent on the codec for that logical bitstream
\end{alltt}\normalsize 


\begin{itemize}
\item Every {\tt ogg\_\-packet} contains a {\em granulepos\/}. The {\em granulepos\/} of each packet is used mostly for {\bf seeking. }{\rm (p.\,\pageref{group__seek__api})}\end{itemize}
\subsection{Structuring}\label{Structuring}
The general structure of an Ogg stream is governed by various rules.\subsubsection{Secondary header packets}\label{secondaries}
Some data sources require initial setup information such as comments and codebooks to be present near the beginning of the stream (directly following the b\_\-o\_\-s packets.

\small\begin{alltt}
   Ogg also allows but does not require secondary header packets after
   the bos page for logical bitstreams and these must also precede any
   data packets in any logical bitstream.  These subsequent header
   packets are framed into an integral number of pages, which will not
   contain any data packets.  So, a physical bitstream begins with the
   bos pages of all logical bitstreams containing one initial header
   packet per page, followed by the subsidiary header packets of all
   streams, followed by pages containing data packets.
\end{alltt}\normalsize 


\begin{itemize}
\item liboggz handles the framing of {\em packets\/} into low-level {\em pages\/}. To ensure that the pages used by secondary headers contain no data packets, set the {\em flush\/} parameter of {\bf oggz\_\-write\_\-feed()}{\rm (p.\,\pageref{group__write__api_a2})} to {\em OGGZ\_\-FLUSH\_\-AFTER\/} when queueing the last of the secondary headers.\item or, equivalently, set {\em flush\/} to {\em OGGZ\_\-FLUSH\_\-BEFORE\/} when queueing the first of the data packets.\end{itemize}
\subsubsection{Sequencing b\_\-o\_\-s and e\_\-o\_\-s packets}\label{boseosseq}
The following rules apply for sequencing {\em bos\/} and {\em eos\/} packets in a physical bitstream: \small\begin{alltt}
   ... All bos pages of all logical bitstreams MUST appear together at
   the beginning of the Ogg bitstream.\end{alltt}\normalsize 


\small\begin{alltt}   ... eos pages for the logical bitstreams need not all occur
   contiguously.  Eos pages may be 'nil' pages, that is, pages
   containing no content but simply a page header with position
   information and the eos flag set in the page header.
\end{alltt}\normalsize 


\begin{itemize}
\item {\bf oggz\_\-write\_\-feed()}{\rm (p.\,\pageref{group__write__api_a2})} will fail with a return value of {\em OGGZ\_\-ERR\_\-BOS\/} if an attempt is made to queue a late {\em bos\/} packet\end{itemize}
\subsubsection{Interleaving logical bitstreams}\label{interleaving}
\small\begin{alltt}
   It is possible to consecutively chain groups of concurrently
   multiplexed bitstreams.  The groups, when unchained, MUST stand on
   their own as a valid concurrently multiplexed bitstream.  The
   following diagram shows a schematic example of such a physical
   bitstream that obeys all the rules of both grouped and chained
   multiplexed bitstreams.\end{alltt}\normalsize 


\small\begin{alltt}               physical bitstream with pages of 
          different logical bitstreams grouped and chained
      -------------------------------------------------------------
      |*A*|*B*|*C*|A|A|C|B|A|B|#A\#|C|...|B|C|#B\#|#C\#|*D*|D|...|#D\#|
      -------------------------------------------------------------
       bos bos bos             eos           eos eos bos       eos\end{alltt}\normalsize 


\small\begin{alltt}   In this example, there are two chained physical bitstreams, the first
   of which is a grouped stream of three logical bitstreams A, B, and C.
   The second physical bitstream is chained after the end of the grouped
   bitstream, which ends after the last eos page of all its grouped
   logical bitstreams.  As can be seen, grouped bitstreams begin
   together - all of the bos pages MUST appear before any data pages.
   It can also be seen that pages of concurrently multiplexed bitstreams
   need not conform to a regular order.  And it can be seen that a
   grouped bitstream can end long before the other bitstreams in the
   group end.
\end{alltt}\normalsize 


\begin{itemize}
\item {\bf oggz\_\-write\_\-feed()}{\rm (p.\,\pageref{group__write__api_a2})} will fail, returning an explicit error value, if an attempt is made to queue a packet in violation of these rules.\end{itemize}
\subsection{References}\label{References}
This introduction to the Ogg format is derived from IETF {\tt RFC 3533} {\em The Ogg File Format version 0\/} in accordance with the following copyright statement pertaining to the text of RFC 3533:

\small\begin{alltt}
   Copyright (C) The Internet Society (2003).  All Rights Reserved.\end{alltt}\normalsize 


\small\begin{alltt}   This document and translations of it may be copied and furnished to
   others, and derivative works that comment on or otherwise explain it
   or assist in its implementation may be prepared, copied, published
   and distributed, in whole or in part, without restriction of any
   kind, provided that the above copyright notice and this paragraph are
   included on all such copies and derivative works.  However, this
   document itself may not be modified in any way, such as by removing
   the copyright notice or references to the Internet Society or other
   Internet organizations, except as needed for the purpose of
   developing Internet standards in which case the procedures for
   copyrights defined in the Internet Standards process must be
   followed, or as required to translate it into languages other than
   English.\end{alltt}\normalsize 


\small\begin{alltt}   The limited permissions granted above are perpetual and will not be
   revoked by the Internet Society or its successors or assigns.\end{alltt}\normalsize 


\small\begin{alltt}   This document and the information contained herein is provided on an
   "AS IS" basis and THE INTERNET SOCIETY AND THE INTERNET ENGINEERING
   TASK FORCE DISCLAIMS ALL WARRANTIES, EXPRESS OR IMPLIED, INCLUDING
   BUT NOT LIMITED TO ANY WARRANTY THAT THE USE OF THE INFORMATION
   HEREIN WILL NOT INFRINGE ANY RIGHTS OR ANY IMPLIED WARRANTIES OF
   MERCHANTABILITY OR FITNESS FOR A PARTICULAR PURPOSE.
\end{alltt}\normalsize 
 

