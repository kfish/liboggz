\section{OGGZ Seek API}
\label{group__seeking__group}\index{OGGZ Seek API@{OGGZ Seek API}}


\subsection{Detailed Description}
The seeking semantics of the Ogg file format were outlined by Monty in {\tt a post to theora-dev} in September 2002. 

Quoting from that post, we have the following assumptions:

\begin{itemize}
\item Ogg is not a non-linear format. ... It is a media transport format designed to do nothing more than deliver content, in a stream, and have all the pieces arrive on time and in sync.\item The Ogg layer does not know the specifics of the codec data it's multiplexing into a stream. It knows nothing beyond 'Oooo, packets!', that the packets belong to different buckets, that the packets go in order, and that packets have position markers. Ogg does not even have a concept of 'time'; it only knows about the sequentially increasing, unitless position markers. It is up to higher layers which have access to the codec APIs to assign and convert units of framing or time.\end{itemize}


(For more details on the structure of Ogg streams, see the {\bf Ogg Basics }{\rm (p.\,\pageref{group__basics})} section).

liboggz provides two abstractions for seeking at an arbitrary level of precision, as well as allowing seeking to a direct byte offset.

To seek across non-metric spaces for which a partial order exists (ie. data that is not synchronised by a measure such as time, but is nevertheless somehow seekably structured), use an Oggz\-Order.

For most purposes, such as media data, use an Oggz\-Metric instead. For common codecs you can ask Oggz to do this for you automatically by instantiating the OGGZ handle with the OGGZ\_\-AUTO flag set. For others you can specify a multiplier with {\bf oggz\_\-set\_\-metric\_\-linear()}{\rm (p.\,\pageref{group__seeking__group_a1})}, or a generic non-linear metric with {\bf oggz\_\-set\_\-metric()}{\rm (p.\,\pageref{group__seeking__group_a2})}.

\begin{itemize}
\item See the {\bf Using Oggz\-Metrics }{\rm (p.\,\pageref{group__metric})} section for details on setting up and seeking with metrics. \end{itemize}


\subsection*{Typedefs}
\begin{CompactItemize}
\item 
typedef ogg\_\-int64\_\-t($\ast$ {\bf Oggz\-Metric} )({\bf OGGZ} $\ast$oggz, long serialno, ogg\_\-int64\_\-t granulepos, void $\ast$user\_\-data)
\begin{CompactList}\small\item\em This is the signature of a function to correlate Ogg streams. \item\end{CompactList}\end{CompactItemize}
\subsection*{Functions}
\begin{CompactItemize}
\item 
int {\bf oggz\_\-set\_\-metric\_\-linear} ({\bf OGGZ} $\ast$oggz, long serialno, ogg\_\-int64\_\-t granule\_\-rate\_\-numerator, ogg\_\-int64\_\-t granule\_\-rate\_\-denominator)
\begin{CompactList}\small\item\em Specify that a logical bitstream has a linear metric. \item\end{CompactList}\item 
int {\bf oggz\_\-set\_\-metric} ({\bf OGGZ} $\ast$oggz, long serialno, {\bf Oggz\-Metric} metric, void $\ast$user\_\-data)
\begin{CompactList}\small\item\em Set the Oggz\-Metric to use for an OGGZ handle. \item\end{CompactList}\item 
ogg\_\-int64\_\-t {\bf oggz\_\-seek\_\-units} ({\bf OGGZ} $\ast$oggz, ogg\_\-int64\_\-t units, int whence)
\begin{CompactList}\small\item\em Seek to a number of units corresponding to the Metric function. \item\end{CompactList}\item 
int {\bf oggz\_\-data\_\-begins\_\-here} ({\bf OGGZ} $\ast$oggz)
\begin{CompactList}\small\item\em Tell OGGZ that we're past the headers, to remember the current packet as the start of data. \item\end{CompactList}\item 
off\_\-t {\bf oggz\_\-tell} ({\bf OGGZ} $\ast$oggz)
\begin{CompactList}\small\item\em Provide the file offset in bytes corresponding to the data read. \item\end{CompactList}\item 
off\_\-t {\bf oggz\_\-seek} ({\bf OGGZ} $\ast$oggz, off\_\-t offset, int whence)
\begin{CompactList}\small\item\em Seek to a specific bytes offset. \item\end{CompactList}\end{CompactItemize}


\subsection{Typedef Documentation}
\index{seeking_group@{seeking\_\-group}!OggzMetric@{OggzMetric}}
\index{OggzMetric@{OggzMetric}!seeking_group@{seeking\_\-group}}
\subsubsection{\setlength{\rightskip}{0pt plus 5cm}typedef ogg\_\-int64\_\-t($\ast$ {\bf Oggz\-Metric})({\bf OGGZ} $\ast$ oggz, long serialno, ogg\_\-int64\_\-t granulepos, void $\ast$ user\_\-data)}\label{group__seeking__group_a0}


This is the signature of a function to correlate Ogg streams. 

If every position in an Ogg stream can be described by a metric (eg. time) then define this function that returns some arbitrary unit value. This is the normal use of OGGZ for media streams. The meaning of units is arbitrary, but must be consistent across all logical bitstreams; for example a conversion of the time offset of a given packet into nanoseconds or a similar stream-specific subdivision may be appropriate.

\begin{Desc}
\item[Parameters:]
\begin{description}
\item[{\em oggz}]An OGGZ handle \item[{\em serialno}]Identifies a logical bitstream within {\em oggz\/} \item[{\em granulepos}]A granulepos within the logical bitstream identified by {\em serialno\/} \item[{\em user\_\-data}]Arbitrary data you wish to pass to your callback \end{description}
\end{Desc}
\begin{Desc}
\item[Returns:]A conversion of the (serialno, granulepos) pair into a measure in units which is consistent across all logical bitstreams within {\em oggz\/} \end{Desc}


\subsection{Function Documentation}
\index{seeking_group@{seeking\_\-group}!oggz_data_begins_here@{oggz\_\-data\_\-begins\_\-here}}
\index{oggz_data_begins_here@{oggz\_\-data\_\-begins\_\-here}!seeking_group@{seeking\_\-group}}
\subsubsection{\setlength{\rightskip}{0pt plus 5cm}int oggz\_\-data\_\-begins\_\-here ({\bf OGGZ} $\ast$ {\em oggz})}\label{group__seeking__group_a4}


Tell OGGZ that we're past the headers, to remember the current packet as the start of data. 

This informs the seeking mechanism that when seeking back to unit 0, go to the packet we're on now, not to the start of the file, which is usually codec headers. \begin{Desc}
\item[Parameters:]
\begin{description}
\item[{\em oggz}]An OGGZ handle previously opened for reading \end{description}
\end{Desc}
\begin{Desc}
\item[Returns:]0 on success, -1 on failure. \end{Desc}
\index{seeking_group@{seeking\_\-group}!oggz_seek@{oggz\_\-seek}}
\index{oggz_seek@{oggz\_\-seek}!seeking_group@{seeking\_\-group}}
\subsubsection{\setlength{\rightskip}{0pt plus 5cm}off\_\-t oggz\_\-seek ({\bf OGGZ} $\ast$ {\em oggz}, off\_\-t {\em offset}, int {\em whence})}\label{group__seeking__group_a6}


Seek to a specific bytes offset. 

\begin{Desc}
\item[Parameters:]
\begin{description}
\item[{\em oggz}]An OGGZ handle \item[{\em offset}]a byte offset \item[{\em whence}]As defined in $<$stdio.h$>$: SEEK\_\-SET, SEEK\_\-CUR or SEEK\_\-END \end{description}
\end{Desc}
\begin{Desc}
\item[Returns:]the new file offset, or -1 on failure. \end{Desc}
\index{seeking_group@{seeking\_\-group}!oggz_seek_units@{oggz\_\-seek\_\-units}}
\index{oggz_seek_units@{oggz\_\-seek\_\-units}!seeking_group@{seeking\_\-group}}
\subsubsection{\setlength{\rightskip}{0pt plus 5cm}ogg\_\-int64\_\-t oggz\_\-seek\_\-units ({\bf OGGZ} $\ast$ {\em oggz}, ogg\_\-int64\_\-t {\em units}, int {\em whence})}\label{group__seeking__group_a3}


Seek to a number of units corresponding to the Metric function. 

\begin{Desc}
\item[Parameters:]
\begin{description}
\item[{\em oggz}]An OGGZ handle \item[{\em units}]A number of units \item[{\em whence}]As defined in $<$stdio.h$>$: SEEK\_\-SET, SEEK\_\-CUR or SEEK\_\-END \end{description}
\end{Desc}
\begin{Desc}
\item[Returns:]the new file offset, or -1 on failure. \end{Desc}
\index{seeking_group@{seeking\_\-group}!oggz_set_metric@{oggz\_\-set\_\-metric}}
\index{oggz_set_metric@{oggz\_\-set\_\-metric}!seeking_group@{seeking\_\-group}}
\subsubsection{\setlength{\rightskip}{0pt plus 5cm}int oggz\_\-set\_\-metric ({\bf OGGZ} $\ast$ {\em oggz}, long {\em serialno}, {\bf Oggz\-Metric} {\em metric}, void $\ast$ {\em user\_\-data})}\label{group__seeking__group_a2}


Set the Oggz\-Metric to use for an OGGZ handle. 

\begin{Desc}
\item[Parameters:]
\begin{description}
\item[{\em oggz}]An OGGZ handle \item[{\em serialno}]Identify the logical bitstream in {\em oggz\/} to attach this metric to. A value of -1 indicates that this metric should be attached to all unattached logical bitstreams in {\em oggz\/}. \item[{\em metric}]An Oggz\-Metric callback \item[{\em user\_\-data}]arbitrary data to pass to the metric callback\end{description}
\end{Desc}
\begin{Desc}
\item[Returns:]0 Success \end{Desc}
\begin{Desc}
\item[Return values:]
\begin{description}
\item[{\em OGGZ\_\-ERR\_\-BAD\_\-SERIALNO}]{\em serialno\/} does not identify an existing logical bitstream in {\em oggz\/}, and is not -1 \item[{\em OGGZ\_\-ERR\_\-BAD\_\-OGGZ}]{\em oggz\/} does not refer to an existing OGGZ\end{description}
\end{Desc}
\begin{Desc}
\item[Note:]Specifying values of {\em serialno\/} other than -1 allows you to pass logical bitstream specific user\_\-data to the same metric. 

Alternatively, you may use a different {\em metric\/} for each {\em serialno\/}, but all metrics used must return mutually consistent unit measurements. \end{Desc}
\index{seeking_group@{seeking\_\-group}!oggz_set_metric_linear@{oggz\_\-set\_\-metric\_\-linear}}
\index{oggz_set_metric_linear@{oggz\_\-set\_\-metric\_\-linear}!seeking_group@{seeking\_\-group}}
\subsubsection{\setlength{\rightskip}{0pt plus 5cm}int oggz\_\-set\_\-metric\_\-linear ({\bf OGGZ} $\ast$ {\em oggz}, long {\em serialno}, ogg\_\-int64\_\-t {\em granule\_\-rate\_\-numerator}, ogg\_\-int64\_\-t {\em granule\_\-rate\_\-denominator})}\label{group__seeking__group_a1}


Specify that a logical bitstream has a linear metric. 

\begin{Desc}
\item[Parameters:]
\begin{description}
\item[{\em oggz}]An OGGZ handle \item[{\em serialno}]Identify the logical bitstream in {\em oggz\/} to attach this linear metric to. A value of -1 indicates that the metric should be attached to all unattached logical bitstreams in {\em oggz\/}. \item[{\em granule\_\-rate\_\-numerator}]The numerator of the granule rate \item[{\em granule\_\-rate\_\-denominator}]The denominator of the granule rate \end{description}
\end{Desc}
\begin{Desc}
\item[Returns:]0 Success \end{Desc}
\begin{Desc}
\item[Return values:]
\begin{description}
\item[{\em OGGZ\_\-ERR\_\-BAD\_\-SERIALNO}]{\em serialno\/} does not identify an existing logical bitstream in {\em oggz\/}. \item[{\em OGGZ\_\-ERR\_\-BAD\_\-OGGZ}]{\em oggz\/} does not refer to an existing OGGZ \end{description}
\end{Desc}
\index{seeking_group@{seeking\_\-group}!oggz_tell@{oggz\_\-tell}}
\index{oggz_tell@{oggz\_\-tell}!seeking_group@{seeking\_\-group}}
\subsubsection{\setlength{\rightskip}{0pt plus 5cm}off\_\-t oggz\_\-tell ({\bf OGGZ} $\ast$ {\em oggz})}\label{group__seeking__group_a5}


Provide the file offset in bytes corresponding to the data read. 

\begin{Desc}
\item[Parameters:]
\begin{description}
\item[{\em oggz}]An OGGZ handle \end{description}
\end{Desc}
\begin{Desc}
\item[Returns:]The current offset of oggz.\end{Desc}
\begin{Desc}
\item[Note:]When reading, the value returned by {\bf oggz\_\-tell()}{\rm (p.\,\pageref{group__seeking__group_a5})} reflects the data offset of the start of the most recent packet processed, so that when called from an Oggz\-Read\-Packet callback it reflects the byte offset of the start of the packet. As OGGZ may have internally read ahead, this may differ from the current offset of the associated file descriptor. \end{Desc}
